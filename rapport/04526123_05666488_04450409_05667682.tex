\documentclass[12pt,letterpaper]{article}
\usepackage{amsmath}
\usepackage{fancyhdr}
\usepackage{graphicx}
\usepackage{alltt}
\usepackage{color}
\usepackage{colortbl}
\usepackage{fullpage}
\usepackage{setspace}
\usepackage{pstricks}
\usepackage{verbatim}
\usepackage{comment}
\usepackage{framed}
\usepackage{listings}
\usepackage{longtable}
\usepackage{pdflscape}
\usepackage{multirow}
\usepackage[config=altsf]{subfig}
\usepackage[utf8]{inputenc}
\usepackage[francais]{babel}
\usepackage[plainpages=false,pdfpagelabels,hypertexnames=false]{hyperref}

%For pdf selection
\usepackage[T1]{fontenc}
\usepackage{lmodern}

%%%%% STYLE %%%%%%%
\topmargin	0in
\topskip	0in
\headheight	0in
\headsep	0in
\parindent	0in
\topsep		0in
\parskip	8pt
\floatsep	0in
%%%%%%%%%%%%%%%%%%%%

%%% SETUP HYPERLINK %%%%%
\hypersetup{
colorlinks 	= true,
linkcolor 	= black}
%%%%%%%%%%%%%%%%%%%%%%%%%

\begin{document}

%%%%%%% COMMANDS %%%%%%%%
\renewcommand{\labelitemi}{$\bullet$}
\newcommand{\unit}[1]{\ \mathrm{#1}}
\newcommand{\degree}{\ensuremath{^\circ}}
%%%%%%%%%%%%%%%%%%%%%%%%%

%%%%%%%%%%%%%%%%%%%%% PAGE TITRE %%%%%%%%%%%%%%%%%%%%%%%%%%%%%%%%%%%%
%%%%%%%%%%%%%%%%%%%%%%%%%%%%%%%%%%%%%%%%%%%%%%%%%%%%%%%%%%%%%%%%%%%%%
\thispagestyle{empty}
\begin{center}
	\vspace{20pt}
	\large{\textsc{
		Intelligence artificielle bio-inspirée\\
	}}
	\vspace{20pt}
	\large{\textsc{
		P02
	}}
	\vfill
	\begin{tabular}{ll}
      Simon Mathieu & 04 450 409 \\
      Steven Denis & 05 667 682 \\
	  Michael Janelle-Montcalm & 04 526 123 \\
	  Martin Provencher &	05 666 488 \\
	\end{tabular}
	\vfill
	Novembre 2009 \\
	\textbf{Université de Sherbrooke}
	\vspace{20pt}
\end{center}
\clearpage
%%%%%%%%%%%%%%%%%%%%% TABLE DES MATIÈRES %%%%%%%%%%%%%%%%%%%%%%%%%%%%
%%%%%%%%%%%%%%%%%%%%%%%%%%%%%%%%%%%%%%%%%%%%%%%%%%%%%%%%%%%%%%%%%%%%%
\begin{spacing}{0.1}
\tableofcontents
\end{spacing}
\clearpage

\section{Introduction}

\section{Analyse des données}

\section{Hypothèses simplificatrices}

\subsection{Logique floue}

\subsection{Réseau de neurones}

\section{Représentation de l'information}

\subsection{Logique floue}

\subsection{Réseau de neurones}

\section{Mise en oeuvre}

\subsection{Logique floue}

\subsection{Réseau de neurones}

\subsection{Algorithme génétique}

Originalement, nous voulious utiliser le nombre de piques dans le graphique de la dérivé pour détecter les chutes.
Si le graphique d'un capteur contenait un pique, il s'agit probablement d'une chute, si il en contient aucun, il 
ne s'agît pas d'une chute. Si il en contient plusieurs, il s'agit d'une récupération de chute. 

Pour détecter un pique, nous utilisions une librairie matlab trouvé sur internet. La fonction de détection de pique
possède 4 paramètres permettant de contrôller quel points du graphique sont détectés comme étant des piques. 

Comme ces paramètre peuvents prendrent une très grand quantité de valuers, il s'agissait d'un cas intéressant ou utiliser
un algorithme génétique pour trouver les valeurs optimales. 

La première étape de la mise en oueuvre fut de manuellement se créer des données d'entrainement. Nous avons donc manuellement
observé les graphiques de la dérivé de certains des capteurs et avons estimé le nombre de pique. 

Armé de ces données, nous avons ensuite définit une fonction de pertinence qui permet de calculer la distance une données est de 
la valeur réel. 

Nous avons ensuite définit une population d'individu qui se composait d'un ensemble de valeurs représentent les paramètre des la
fonction qui trouve les piques. 

La prochaine étape consiste a faire reproduire et muter les individus de notre population pour produire la prochaine génération. Après 
observation, nous avons conclus qu'il était mieux de conserver les individus d'un génération dans la prochaine. Nous gardons donc les 
cinq individus les mieux adapter. 

Le choix des individus qui se reproduisent est fait à l'aide d'une fonction aléatoire pondéré de façon a ce que les individus possèdant
des meilleurs gènes aillent une meilleur probabilité de se reproduire. 

Le speudo code de notre algorithme est:

\begin{verbatim}
pop = InitPopulation

FOR n generation
  fitenesses = CalculateFitnesses pop
  breeders = SelectBreeders pop
  pop = Reproduce breeders
  pop = Mutate pop

\end{verbatim}


\section{Évaluation des performances}

\subsection{Logique floue}

\subsection{Réseau de neurones}

\subsection{Algorithme génétique}

Dans notre cas, l'algorithme générique nous a permis de converger assé rapidement vers une solution assé optmiale. 

Malheureusement, trouver le nombre de piques dans une fonction n'est pas un problème simple. La librairie que nous utilisions 
s'est avérer insufisente pour être capable de détecter les pics correctement dans les données que nous avions. 

% Martin est-ce que tu peux comléter avec les paramètre utilisé pour entrainer l'algo et le nombre de générations nécessaires.
% Tu n'as pas commiter ces infos sur git.

\section{Observations et perspectives futures}

\end{document}
