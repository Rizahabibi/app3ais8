\documentclass[12pt,letterpaper]{article}
\usepackage{amsmath}
\usepackage{fancyhdr}
\usepackage{graphicx}
\usepackage{alltt}
\usepackage{color}
\usepackage{colortbl}
\usepackage{fullpage}
\usepackage{setspace}
\usepackage{pstricks}
\usepackage{verbatim}
\usepackage{comment}
\usepackage{framed}
\usepackage{listings}
\usepackage{longtable}
\usepackage{pdflscape}
\usepackage{multirow}
\usepackage[config=altsf]{subfig}
\usepackage[utf8]{inputenc}
\usepackage[francais]{babel}
\usepackage[plainpages=false,pdfpagelabels,hypertexnames=false]{hyperref}

%For pdf selection
\usepackage[T1]{fontenc}
\usepackage{lmodern}

%%%%% STYLE %%%%%%%
\topmargin	0in
\topskip	0in
\headheight	0in
\headsep	0in
\parindent	0in
\topsep		0in
\parskip	8pt
\floatsep	0in
%%%%%%%%%%%%%%%%%%%%

%%% SETUP HYPERLINK %%%%%
\hypersetup{
colorlinks 	= true,
linkcolor 	= black}
%%%%%%%%%%%%%%%%%%%%%%%%%

\begin{document}

%%%%%%% COMMANDS %%%%%%%%
\renewcommand{\labelitemi}{$\bullet$}
\newcommand{\unit}[1]{\ \mathrm{#1}}
\newcommand{\degree}{\ensuremath{^\circ}}
%%%%%%%%%%%%%%%%%%%%%%%%%

%%%%%%%%%%%%%%%%%%%%% PAGE TITRE %%%%%%%%%%%%%%%%%%%%%%%%%%%%%%%%%%%%
%%%%%%%%%%%%%%%%%%%%%%%%%%%%%%%%%%%%%%%%%%%%%%%%%%%%%%%%%%%%%%%%%%%%%
\thispagestyle{empty}
\begin{center}
	\vspace{20pt}
	\large{\textsc{
		Intelligence artificielle bio-inspirée\\
	}}
	\vspace{20pt}
	\large{\textsc{
		P02
	}}
	\vfill
	\begin{tabular}{ll}
      Simon Mathieu & 04 450 409 \\
      Steven Denis & 05 667 682 \\
	  Michael Janelle-Montcalm & 04 526 123 \\
	  Martin Provencher &	05 666 488 \\
	\end{tabular}
	\vfill
	Novembre 2009 \\
	\textbf{Université de Sherbrooke}
	\vspace{20pt}
\end{center}
\clearpage
%%%%%%%%%%%%%%%%%%%%% TABLE DES MATIÈRES %%%%%%%%%%%%%%%%%%%%%%%%%%%%
%%%%%%%%%%%%%%%%%%%%%%%%%%%%%%%%%%%%%%%%%%%%%%%%%%%%%%%%%%%%%%%%%%%%%
\begin{spacing}{0.1}
\tableofcontents
\end{spacing}
\clearpage

\section{Introduction} % Steven
La chute chez les personnes agées représente une cause importante d'hospitalisation chez ce groupe de personnes. C'est pour cette raison que le Centre de recherche sur le vieillissement de l'Institut de gériatrie de l'Université de Sherbrooke tente de mettre sur pied un système de détection de chutes. Pour ce faire, ce Centre nous a demandé de réaliser une étude de faisabilité qui évalue plusieurs techniques d'intelligence artificielle capable d'identifier l'occurence d'une chute. Le rapport qui suit présentera l'analyse des données effectuée avant le traitement et l'étude des techniques d'intelligence artifielle. À la suite de l'analyse, nous verrons les hypothèses simplificatrices que nous avons établies pour évaluer chacune des techniques. Ensuite, il y aura une représentation de l'information, la mise en oeuvre, et l'évaluation des différentes techniques à l'étude. Pour terminer, nous noterons les observations effectuées ainsi que les perspectives futures.

\section{Analyse des données} % Steven
Afin d'émettre des hypothèses simplificatrices, nous avons d'abord effectué une analyse des données qui nous ont été fourni pour faire l'étude des techniques pouvant être utilisées pour détecter l'occurence d'une chute.

% Similitudes entre les canaux (redondance entre 1 et 6, on garde 6)
% Valeurs des maxima et minima sont plus grandes lors d'une chute que lors
% d'une non-chute

\section{Hypothèses simplificatrices}

\subsection{Logique floue}

\subsection{Réseau de neurones} % Mike
% Utilisation seulement des max et min permet d'identifier les chutes

\section{Représentation de l'information}

\subsection{Logique floue}

\subsection{Réseau de neurones} % Steven

% Schéma-bloc du système
% Extraction des caractéristiques (max, min)
% Entrées et sorties (fichiers de données, sorties du script)

\section{Mise en oeuvre}

\subsection{Logique floue}

\subsection{Réseau de neurones} % Mike
% Données d'entraînement (sujet 5)
% Description de l'évolution du réseau (époques)
% Paramètres d'entraînement (momentum, learning rate)
% Ne converge pas toujours
% – Loi d’apprentissage, nombre d’unités cachées, nombre d’unités de sortie a expérimenter ;
% – Critères d’entraînement et d’évaluation ;
% – Création des ensembles d’entraînement et de test en lien avec l’apprentissage ;
% – Critère de classification et de reconnaissance.

\subsection{Algorithme génétique}

Originalement, nous voulious utiliser le nombre de piques dans le graphique de la dérivé pour détecter les chutes.
Si le graphique d'un capteur contenait un pique, il s'agit probablement d'une chute, si il en contient aucun, il
ne s'agît pas d'une chute. Si il en contient plusieurs, il s'agit d'une récupération de chute.

Pour détecter un pique, nous utilisions une librairie matlab trouvé sur internet. La fonction de détection de pique
possède 4 paramètres permettant de contrôller quel points du graphique sont détectés comme étant des piques.

Comme ces paramètre peuvents prendrent une très grand quantité de valuers, il s'agissait d'un cas intéressant ou utiliser
un algorithme génétique pour trouver les valeurs optimales.

La première étape de la mise en oueuvre fut de manuellement se créer des données d'entrainement. Nous avons donc manuellement
observé les graphiques de la dérivé de certains des capteurs et avons estimé le nombre de pique.

Armé de ces données, nous avons ensuite définit une fonction de pertinence qui permet de calculer la distance une données est de
la valeur réel.

Nous avons ensuite définit une population d'individu qui se composait d'un ensemble de valeurs représentent les paramètre des la
fonction qui trouve les piques.

La prochaine étape consiste a faire reproduire et muter les individus de notre population pour produire la prochaine génération. Après
observation, nous avons conclus qu'il était mieux de conserver les individus d'un génération dans la prochaine. Nous gardons donc les
cinq individus les mieux adapter.

Le choix des individus qui se reproduisent est fait à l'aide d'une fonction aléatoire pondéré de façon a ce que les individus possèdant
des meilleurs gènes aillent une meilleur probabilité de se reproduire.

Le speudo code de notre algorithme est:

\begin{verbatim}
pop = InitPopulation

FOR n generation
  fitenesses = CalculateFitnesses pop
  breeders = SelectBreeders pop
  pop = Reproduce breeders
  pop = Mutate pop

\end{verbatim}


\section{Évaluation des performances}

\subsection{Logique floue}

\subsection{Réseau de neurones} % Mike
% Taux d'identification (avec explications des variations selon les paramètres)
% Performance avec les données d'entraînement, puis les données de test
% Différence des résultats selon le nombre d'unités cachées

\subsection{Algorithme génétique}

Dans notre cas, l'algorithme générique nous a permis de converger assé rapidement vers une solution assé optmiale.

Malheureusement, trouver le nombre de piques dans une fonction n'est pas un problème simple. La librairie que nous utilisions
s'est avérer insufisente pour être capable de détecter les pics correctement dans les données que nous avions.

% Martin est-ce que tu peux comléter avec les paramètre utilisé pour entrainer l'algo et le nombre de générations nécessaires.
% Tu n'as pas commiter ces infos sur git.

\section{Observations et perspectives futures}

\end{document}
